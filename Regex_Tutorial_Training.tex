\PassOptionsToPackage{unicode=true}{hyperref} % options for packages loaded elsewhere
\PassOptionsToPackage{hyphens}{url}
%
\documentclass[ignorenonframetext,]{beamer}
\usepackage{pgfpages}
\setbeamertemplate{caption}[numbered]
\setbeamertemplate{caption label separator}{: }
\setbeamercolor{caption name}{fg=normal text.fg}
\beamertemplatenavigationsymbolsempty
% Prevent slide breaks in the middle of a paragraph:
\widowpenalties 1 10000
\raggedbottom
\setbeamertemplate{part page}{
\centering
\begin{beamercolorbox}[sep=16pt,center]{part title}
  \usebeamerfont{part title}\insertpart\par
\end{beamercolorbox}
}
\setbeamertemplate{section page}{
\centering
\begin{beamercolorbox}[sep=12pt,center]{part title}
  \usebeamerfont{section title}\insertsection\par
\end{beamercolorbox}
}
\setbeamertemplate{subsection page}{
\centering
\begin{beamercolorbox}[sep=8pt,center]{part title}
  \usebeamerfont{subsection title}\insertsubsection\par
\end{beamercolorbox}
}
\AtBeginPart{
  \frame{\partpage}
}
\AtBeginSection{
  \ifbibliography
  \else
    \frame{\sectionpage}
  \fi
}
\AtBeginSubsection{
  \frame{\subsectionpage}
}
\usepackage{lmodern}
\usepackage{amssymb,amsmath}
\usepackage{ifxetex,ifluatex}
\usepackage{fixltx2e} % provides \textsubscript
\ifnum 0\ifxetex 1\fi\ifluatex 1\fi=0 % if pdftex
  \usepackage[T1]{fontenc}
  \usepackage[utf8]{inputenc}
  \usepackage{textcomp} % provides euro and other symbols
\else % if luatex or xelatex
  \usepackage{unicode-math}
  \defaultfontfeatures{Ligatures=TeX,Scale=MatchLowercase}
\fi
% use upquote if available, for straight quotes in verbatim environments
\IfFileExists{upquote.sty}{\usepackage{upquote}}{}
% use microtype if available
\IfFileExists{microtype.sty}{%
\usepackage[]{microtype}
\UseMicrotypeSet[protrusion]{basicmath} % disable protrusion for tt fonts
}{}
\IfFileExists{parskip.sty}{%
\usepackage{parskip}
}{% else
\setlength{\parindent}{0pt}
\setlength{\parskip}{6pt plus 2pt minus 1pt}
}
\usepackage{hyperref}
\hypersetup{
            pdftitle={REGULAR EXPRESSION(REGEX) INTRODUCTION},
            pdfborder={0 0 0},
            breaklinks=true}
\urlstyle{same}  % don't use monospace font for urls
\newif\ifbibliography
\usepackage{color}
\usepackage{fancyvrb}
\newcommand{\VerbBar}{|}
\newcommand{\VERB}{\Verb[commandchars=\\\{\}]}
\DefineVerbatimEnvironment{Highlighting}{Verbatim}{commandchars=\\\{\}}
% Add ',fontsize=\small' for more characters per line
\usepackage{framed}
\definecolor{shadecolor}{RGB}{248,248,248}
\newenvironment{Shaded}{\begin{snugshade}}{\end{snugshade}}
\newcommand{\AlertTok}[1]{\textcolor[rgb]{0.94,0.16,0.16}{#1}}
\newcommand{\AnnotationTok}[1]{\textcolor[rgb]{0.56,0.35,0.01}{\textbf{\textit{#1}}}}
\newcommand{\AttributeTok}[1]{\textcolor[rgb]{0.77,0.63,0.00}{#1}}
\newcommand{\BaseNTok}[1]{\textcolor[rgb]{0.00,0.00,0.81}{#1}}
\newcommand{\BuiltInTok}[1]{#1}
\newcommand{\CharTok}[1]{\textcolor[rgb]{0.31,0.60,0.02}{#1}}
\newcommand{\CommentTok}[1]{\textcolor[rgb]{0.56,0.35,0.01}{\textit{#1}}}
\newcommand{\CommentVarTok}[1]{\textcolor[rgb]{0.56,0.35,0.01}{\textbf{\textit{#1}}}}
\newcommand{\ConstantTok}[1]{\textcolor[rgb]{0.00,0.00,0.00}{#1}}
\newcommand{\ControlFlowTok}[1]{\textcolor[rgb]{0.13,0.29,0.53}{\textbf{#1}}}
\newcommand{\DataTypeTok}[1]{\textcolor[rgb]{0.13,0.29,0.53}{#1}}
\newcommand{\DecValTok}[1]{\textcolor[rgb]{0.00,0.00,0.81}{#1}}
\newcommand{\DocumentationTok}[1]{\textcolor[rgb]{0.56,0.35,0.01}{\textbf{\textit{#1}}}}
\newcommand{\ErrorTok}[1]{\textcolor[rgb]{0.64,0.00,0.00}{\textbf{#1}}}
\newcommand{\ExtensionTok}[1]{#1}
\newcommand{\FloatTok}[1]{\textcolor[rgb]{0.00,0.00,0.81}{#1}}
\newcommand{\FunctionTok}[1]{\textcolor[rgb]{0.00,0.00,0.00}{#1}}
\newcommand{\ImportTok}[1]{#1}
\newcommand{\InformationTok}[1]{\textcolor[rgb]{0.56,0.35,0.01}{\textbf{\textit{#1}}}}
\newcommand{\KeywordTok}[1]{\textcolor[rgb]{0.13,0.29,0.53}{\textbf{#1}}}
\newcommand{\NormalTok}[1]{#1}
\newcommand{\OperatorTok}[1]{\textcolor[rgb]{0.81,0.36,0.00}{\textbf{#1}}}
\newcommand{\OtherTok}[1]{\textcolor[rgb]{0.56,0.35,0.01}{#1}}
\newcommand{\PreprocessorTok}[1]{\textcolor[rgb]{0.56,0.35,0.01}{\textit{#1}}}
\newcommand{\RegionMarkerTok}[1]{#1}
\newcommand{\SpecialCharTok}[1]{\textcolor[rgb]{0.00,0.00,0.00}{#1}}
\newcommand{\SpecialStringTok}[1]{\textcolor[rgb]{0.31,0.60,0.02}{#1}}
\newcommand{\StringTok}[1]{\textcolor[rgb]{0.31,0.60,0.02}{#1}}
\newcommand{\VariableTok}[1]{\textcolor[rgb]{0.00,0.00,0.00}{#1}}
\newcommand{\VerbatimStringTok}[1]{\textcolor[rgb]{0.31,0.60,0.02}{#1}}
\newcommand{\WarningTok}[1]{\textcolor[rgb]{0.56,0.35,0.01}{\textbf{\textit{#1}}}}
\usepackage{longtable,booktabs}
\usepackage{caption}
% These lines are needed to make table captions work with longtable:
\makeatletter
\def\fnum@table{\tablename~\thetable}
\makeatother
\usepackage{graphicx,grffile}
\makeatletter
\def\maxwidth{\ifdim\Gin@nat@width>\linewidth\linewidth\else\Gin@nat@width\fi}
\def\maxheight{\ifdim\Gin@nat@height>\textheight\textheight\else\Gin@nat@height\fi}
\makeatother
% Scale images if necessary, so that they will not overflow the page
% margins by default, and it is still possible to overwrite the defaults
% using explicit options in \includegraphics[width, height, ...]{}
\setkeys{Gin}{width=\maxwidth,height=\maxheight,keepaspectratio}
\setlength{\emergencystretch}{3em}  % prevent overfull lines
\providecommand{\tightlist}{%
  \setlength{\itemsep}{0pt}\setlength{\parskip}{0pt}}
\setcounter{secnumdepth}{0}

% set default figure placement to htbp
\makeatletter
\def\fps@figure{htbp}
\makeatother


\title{REGULAR EXPRESSION(REGEX) INTRODUCTION}
\date{}

\begin{document}
\frame{\titlepage}

\begin{frame}{1. Introduction}
\protect\hypertarget{introduction}{}

What is \textbf{Regular Expression} \emph{a.k.a} \textbf{RegEx} , an
attempt to define this is at times makes the tool use under explained or
overly complicated. Though let me try \textbf{RegEx} in simple terms is
a way of handling and manipulating characters in ways different ways
while simplying the effort of the process.

\end{frame}

\begin{frame}

We use \textbf{RegEx} on a daily basis it is just that we do not know
areas where we use are below:

When using search and find

\begin{itemize}
\tightlist
\item
  Most of us have searched words in word documents excel, the process in
  which this applications are able to search apply \textbf{RegEx}
  expressions at the background
\end{itemize}

When creating passwords

\begin{itemize}
\tightlist
\item
  We find ourselves when creating passwords being told we need special
  characters, capitalized letter, numerics and lower case letters. How
  does the system know when you have not been able to put all this into
  the password. Very thought provocative I would say.
\end{itemize}

\end{frame}

\begin{frame}[fragile]{2. Tenents of RegEx}
\protect\hypertarget{tenents-of-regex}{}

\textbf{RegEx} really on key concepts that make it work:

\begin{longtable}[]{@{}ll@{}}
\toprule
\begin{minipage}[b]{0.28\columnwidth}\raggedright
Rule\strut
\end{minipage} & \begin{minipage}[b]{0.66\columnwidth}\raggedright
Explanation\strut
\end{minipage}\tabularnewline
\midrule
\endhead
\begin{minipage}[t]{0.28\columnwidth}\raggedright
Character Matching\strut
\end{minipage} & \begin{minipage}[t]{0.66\columnwidth}\raggedright
- Being able to locating alphabetic characters e.g.
\texttt{ABCDEF..}\strut
\end{minipage}\tabularnewline
\begin{minipage}[t]{0.28\columnwidth}\raggedright
Numerical Matching\strut
\end{minipage} & \begin{minipage}[t]{0.66\columnwidth}\raggedright
- Locating numerals e.g. \texttt{123-456}\strut
\end{minipage}\tabularnewline
\begin{minipage}[t]{0.28\columnwidth}\raggedright
Special Character Matching\strut
\end{minipage} & \begin{minipage}[t]{0.66\columnwidth}\raggedright
- Locating special characters e.g. \texttt{\$\#@!}\strut
\end{minipage}\tabularnewline
\bottomrule
\end{longtable}

\end{frame}

\begin{frame}[fragile]

\begin{block}{i) Character Matching}

What happens when we want to find a certain character in a text. Lets
get our hands dirty.

\begin{Shaded}
\begin{Highlighting}[]
\NormalTok{ex_text <-}\StringTok{ }\KeywordTok{c}\NormalTok{(}\StringTok{"The"}\NormalTok{,}\StringTok{"fat"}\NormalTok{,}\StringTok{"cat"}\NormalTok{,}\StringTok{"sat on the mat"}\NormalTok{)}
\KeywordTok{grep}\NormalTok{(}\StringTok{"at"}\NormalTok{,ex_text, }\DataTypeTok{value =} \OtherTok{TRUE}\NormalTok{)}
\end{Highlighting}
\end{Shaded}

\begin{verbatim}
## [1] "fat"            "cat"            "sat on the mat"
\end{verbatim}

What you will notice the output picks words that have \texttt{at} any
where on the string vector \texttt{ex\_text}

Am a proponent of \href{http://https://www.tidyverse.org}{Tidyverse}
packages, this are the likes of \emph{tidyr, dplyr, stringr etc}

\end{block}

\end{frame}

\begin{frame}[fragile]

One of my favourite stingr manipulation library is
\href{https://stringr.tidyverse.org/}{stringr}, it has easy
understanding of syntax. Let use the example we had previously

\begin{Shaded}
\begin{Highlighting}[]
\KeywordTok{library}\NormalTok{(stringr)}

\NormalTok{ex_text <-}\StringTok{ }\KeywordTok{c}\NormalTok{(}\StringTok{"The"}\NormalTok{,}\StringTok{"fat"}\NormalTok{,}\StringTok{"cat"}\NormalTok{,}\StringTok{"sat on the mat"}\NormalTok{)}
\KeywordTok{str_detect}\NormalTok{(ex_text,}\StringTok{"at"}\NormalTok{)}
\end{Highlighting}
\end{Shaded}

\begin{verbatim}
## [1] FALSE  TRUE  TRUE  TRUE
\end{verbatim}

\begin{Shaded}
\begin{Highlighting}[]
\NormalTok{ex_text[}\KeywordTok{str_detect}\NormalTok{(ex_text,}\StringTok{"at"}\NormalTok{)]}
\end{Highlighting}
\end{Shaded}

\begin{verbatim}
## [1] "fat"            "cat"            "sat on the mat"
\end{verbatim}

You will notice from \texttt{str\_detect(ex\_text,"at")} that we get
logical values \texttt{TRUE} shows elements in the vector that match
will \texttt{FALSE} is the opposite.

\end{frame}

\begin{frame}[fragile]

But at times we want to pick alphabetic characters from a text with
mixed characters. How can we do this? Before we talk about that I will
introduce \texttt{meta\ characters} that are very important.

\begin{longtable}[]{@{}ll@{}}
\toprule
\begin{minipage}[b]{0.39\columnwidth}\raggedright
Meta Character\strut
\end{minipage} & \begin{minipage}[b]{0.55\columnwidth}\raggedright
Explanation\strut
\end{minipage}\tabularnewline
\midrule
\endhead
\begin{minipage}[t]{0.39\columnwidth}\raggedright
\texttt{.}\strut
\end{minipage} & \begin{minipage}[t]{0.55\columnwidth}\raggedright
A period is used to connote any instance of a single character\strut
\end{minipage}\tabularnewline
\begin{minipage}[t]{0.39\columnwidth}\raggedright
\texttt{{[}{]}}\strut
\end{minipage} & \begin{minipage}[t]{0.55\columnwidth}\raggedright
Square brackets are used to give ranges both Alphabets and
Numerics\strut
\end{minipage}\tabularnewline
\begin{minipage}[t]{0.39\columnwidth}\raggedright
\texttt{*}\strut
\end{minipage} & \begin{minipage}[t]{0.55\columnwidth}\raggedright
Match multiple characters 0 or more times\strut
\end{minipage}\tabularnewline
\begin{minipage}[t]{0.39\columnwidth}\raggedright
\texttt{+}\strut
\end{minipage} & \begin{minipage}[t]{0.55\columnwidth}\raggedright
Match multiple characters 1 or more times\strut
\end{minipage}\tabularnewline
\begin{minipage}[t]{0.39\columnwidth}\raggedright
\texttt{?}\strut
\end{minipage} & \begin{minipage}[t]{0.55\columnwidth}\raggedright
The next character or instance is optional\strut
\end{minipage}\tabularnewline
\begin{minipage}[t]{0.39\columnwidth}\raggedright
\texttt{\{n,m\}}\strut
\end{minipage} & \begin{minipage}[t]{0.55\columnwidth}\raggedright
Matches a character at least \texttt{n} times and not more than
\texttt{m} times\strut
\end{minipage}\tabularnewline
\begin{minipage}[t]{0.39\columnwidth}\raggedright
\texttt{(xyz)}\strut
\end{minipage} & \begin{minipage}[t]{0.55\columnwidth}\raggedright
Match the exact characters\strut
\end{minipage}\tabularnewline
\begin{minipage}[t]{0.39\columnwidth}\raggedright
\texttt{\textbar{}}\strut
\end{minipage} & \begin{minipage}[t]{0.55\columnwidth}\raggedright
Defacto logic for \texttt{or} pick any\strut
\end{minipage}\tabularnewline
\begin{minipage}[t]{0.39\columnwidth}\raggedright
\texttt{\textbackslash{}}\strut
\end{minipage} & \begin{minipage}[t]{0.55\columnwidth}\raggedright
Escaping characters, when you want to mean special characters are part
of the text and this special characters are part of \textbf{RegEx}
commands\strut
\end{minipage}\tabularnewline
\begin{minipage}[t]{0.39\columnwidth}\raggedright
\texttt{\^{}}\strut
\end{minipage} & \begin{minipage}[t]{0.55\columnwidth}\raggedright
Matches the start of a statement or Word\strut
\end{minipage}\tabularnewline
\begin{minipage}[t]{0.39\columnwidth}\raggedright
\texttt{\$}\strut
\end{minipage} & \begin{minipage}[t]{0.55\columnwidth}\raggedright
Matches the end of a statement or word\strut
\end{minipage}\tabularnewline
\bottomrule
\end{longtable}

\end{frame}

\begin{frame}[fragile]

This is an \href{http://rmarkdown.rstudio.com}{R Markdown} Notebook.
When you execute code within the notebook, the results appear beneath
the code.

Try executing this chunk by clicking the \emph{Run} button within the
chunk or by placing your cursor inside it and pressing
\emph{Ctrl+Shift+Enter}.

\begin{Shaded}
\begin{Highlighting}[]
\KeywordTok{plot}\NormalTok{(cars)}
\end{Highlighting}
\end{Shaded}

\includegraphics{Regex_Tutorial_Training_files/figure-beamer/unnamed-chunk-3-1.pdf}

Add a new chunk by clicking the \emph{Insert Chunk} button on the
toolbar or by pressing \emph{Ctrl+Alt+I}.

When you save the notebook, an HTML file containing the code and output
will be saved alongside it (click the \emph{Preview} button or press
\emph{Ctrl+Shift+K} to preview the HTML file).

The preview shows you a rendered HTML copy of the contents of the
editor. Consequently, unlike \emph{Knit}, \emph{Preview} does not run
any R code chunks. Instead, the output of the chunk when it was last run
in the editor is displayed.

\end{frame}

\begin{frame}

\end{frame}

\end{document}
